\section{A Survey on Theories and Applications for Self-Driving Cars}

https://www.mdpi.com/2076-3417/10/8/2749/htm#B92-applsci-10-02749

\cite{}
Because many self-driving cars technologies rely on image feature representation, they can be easily realized based on CNNs, such as obstacle detection, scene classification, and lane recognition.

* Object detection

Table with datasets - GOOD
UnrealDataset dataset.
KITTI dataset
DUSD dataset

The Málaga Stereo and Laser Urban Data Set
https://www.mrpt.org/MalagaUrbanDataset
MSVUD

DUSD
Daimler Urban Segmentation Dataset
http://www.6d-vision.com/scene-labeling


* Scene Classification and Understanding

 The instantaneity, robustness, and accuracy of road scene recognition are affected by multiple factors such as light variation, complex road environment and severe weather
 
* Lane Recognition 

* Other Applications
In addition to the applications mentioned above, there are many other applications of deep learning methods in self-driving cars, such as 
    path planning, 
    motion control, 
    pedestrian detection, and 
    traffic sign and light detection.
    
There are a number of tasks related to autonomous-vehicle decision-making. An autonomous vehicle must be able to deal with obstacle detection, scene classification, lane recognition, path planning, motion control, pedestrian detection, traffic signage detection (including traffic lights). The list is daunting and by no means exhaustive. Still this research field is thriving, with automakers such as Tesla, Nissan, Audi, General Motors, BMW, Ford, Honda, Toyota, Mercedes and Volkswagen, and technology companies such as Apple, Google, NVidia and Intel, actively researching \cite{app10082749}.
