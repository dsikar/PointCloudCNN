%% \subsubsection{Evaluation}

\subsection{Evaluation}
\label{Evaluation}

A metric is required for computing how close two point clouds are from each other. In our case, a point cloud generated by LIDAR to be compared with a point cloud generated by our model.  
We examined the \textit{mean squared error} (MSE), \textit{intersection over union} (IoU), \textit{mean cross entropy} (CE) \textit{loss} and \textit{chamfer distance} (CD) which we found the most suitable.  

\subsubsection{Chamfer distance (CD)}

Using the definition given by \cite{fan2016point}, we represent the point cloud as the unordered set in $\mathbb{R}^3$, $S = {(x_i,y_i,z_i)}_{i=1}^N$ where N is a predefined constant. $N=1024$ is considered to be sufficient to preserve the major structures.  
We compute a loss $L(S^{pred},S^{gt})$ between the point cloud prediction and ground truth:
\begin{equation}
\label{cd-loss}
L(S^{pred},S^{gt})  = \sum d(S^{pred},S^{gt})
\end{equation}
where $d(S^{pred},S^{gt})$ is the chamfer distance:
\begin{equation}
    d_{CD}(S_1,S_2) = \sum_{x \in S_1} \min_{y \in S_2} ||x-y||_2^2 + \sum_{y \in S_2} \min_{x \in S_1} ||x-y||_2^2
\end{equation}
We omit the $i$ indexes given by the authors in equation \ref{cd-loss} because we use the loss function to compute the minimum distance between only one output (from our model) and ground truth (from LIDAR). Additionally, we define a threshold $T_{loss}$ such that if the computed loss is below threshold, the point cloud generated by the model is deemed to be acceptable and included in the processing pipeline, otherwise, the generated model is discarded and only the LIDAR ground truth is used for further processing.  
There are other aspects including scale that must be considered at a future stage.