\subsection{Mark Allocation}

\textit{In addition, to pass the assessment the single 'pass/fail' criterion that relates to Ethical, Legal and Professional Issues must be achieved.}
REMOVE END

vii. consider the \textit{ethical, legal and professional issues} that are raised by the work that you plan and describe ways in which you intend addressing these issues effectively and comprehensively; 
Reference to the Research Governance Framework \parencite{CSREC:2020}

\subsection{Ethical, Legal \& Professional Issues}

All proposals must make reference to the Research Governance Framework established by the Department of Computer Science Research Ethics Committee (CSREC) and clearly describe any ethical issues that might occur in line with this advice.
All project proposals must include a completed copy of the Research Ethics Form:   https://moodle.city.ac.uk/mod/resource/view.php?id=1244668 
The form has two parts:

PART A: Ethics Checklist. All students must complete this part.  The checklist identifies whether the project requires ethical approval and, if so, where to apply for approval.


PART B: Ethics Proportionate Review Form. Students who have answered “no” to questions 1 – 18 and “yes” to question 19 in the ethics checklist must complete this part. The project supervisor has delegated authority to provide approval in such cases that are considered to involve MINIMAL risk. 
The approval may be provisional: the student may need to seek additional approval from the supervisor as the project progresses and details are established. If not, students will need to apply to CSREC through Research Ethics Online.  We should be able to conduct good research with minimal risk.
                       
Approval takes place after the Proposal has been marked.  

Discussion of the issues that are raised and ways in which they will be addressed should be included within the main body of the proposal under 'Approaches' to demonstrate capabilities in dealing with ethical issues in the RMPI assessment. 

\section*{References}
We strongly recommend using some of the excellent texts that have been written to support students when thinking about project topics, objectives, methods and deliverables.

You should use these in particular to identify and document appropriate research methods.

We particularly recommend:

Oates, B. J. (2006). Researching Information Systems and Computing. London: Sage Publications Ltd, 341pp.
Chapters 1-3 are particularly good at contextualising research, selecting research topics and determining outcomes. The chapters that follow give guidance on particular research methods and references to additional reading that can inform your work. If you are planning a 'Design \& Build' project in which you develop software then please read page 9 of this book and use Chapter 8 'Design and Creation' to inform your approach.

Dawson, C. W. (2009). Projects in Computing and Information Systems: A Student's Guide (2nd ed.). London: Addison Wesley, 304pp.
An updated third edition will soon be available. In the second edition, Chapter 3 provides good guidance on project selection and proposal writing, Chapter 4 is good on Risk Management, whilst Chapter 6 details approaches that should be considered for software development projects.
The new edition of the book is also available in digital form and recommended. 

Those considering using online methods should consult the Exploring Online Research Methods resources:

University of Leicester (2010). Exploring Online Research Methods. 
http://www.restore.ac.uk/orm/
This excellent website provides plenty of resources to learn about using online methods effectively - the Self-Study area is very useful.

You will find useful guidance on citation and use of literature in the RMPI Effective Use of Literature reference list. (See Week 02) Search for the Pears and Shields (2010) book on citation and information about reference management software that we recommend as a means of managing references and producing reference lists.

All references must be reported in a comprehensive bibliography and cited using Harvard of a similar structured referencing system.